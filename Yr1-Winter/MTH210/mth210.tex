\documentclass[a4paper]{article}
\input{pre}

\title{MTH210 | Discrete Mathematics II}
\author{James Li | 501022159 \and Professor: M. Delcourt\and Email: mdelcourt@torontomu.ca}
\date{}

\renewcommand{\contentsname}{\centering Content by Week}

\begin{document}
  \maketitle
  \tableofcontents
  \newpage
  \section{Sequences and Series} 
    A \textbf{sequence} is an ordered set of numbers.
    \begin{displaymath}
      \begin{split}
        &2,4,6,8,10\dots \textrm{ is an example of a sequence of positive numbers.} \\
        &a_1,a_2,a_3,\dots,a_n \textrm{ denotes an infinite sequence.}
      \end{split}
    \end{displaymath}
    \begin{itemize}
      \item A sequence is defined \textbf{analytically} if each term $a_i$ is defined by some function $f(i) = a_i$
      \item A sequence is defined \textbf{recursively} if the first $k$ terms are given \textbf{explicitly} and the rest are given through a recursive function $a_n = f(a_{n-1},a_{n-2},\dots,a_{n-k})$ for $n >k$.
      \item Even if two sequences are equal for small indexes, does not indicate that they don't diverge at some further point.
    \end{itemize}
    A \textbf{series} is the sum of all the terms in a sequence.
    \begin{displaymath}
      \textrm{If $m$ and $n$ are integers and $m \le n$, the symbol $\sum_{k=m}^n a_k$ is the summation from $k$, defined as:}
    \end{displaymath}
     $$ 
        \sum_{k=m}^n a_k = a_m+ a_{m+1}+ a_{m+2} + \dots + a_{n}
     $$
    \begin{itemize}
      \item We call $k$ the \textbf{index} of the summation.
      \item $m$ is the \textbf{lower limit} of the summation.
      \item $n$ is the \textbf{upper limit} of the summation.
    \end{itemize}

    \subsection{Sums and Products}
    If $m$ and $n$ are integers and $m \le n$, the symbol $\prod_{k=m}^n a_k$ is read as product from $k$ equals $m$ to $n$ of $a$ sub $k$, it can be written as:
    $$
    \prod_{k=m}^n a_k = a_m \cdot a_{m+1} \cdot a_{m+2} \times \dots \times a_n
    $$
    \begin{theorem}
      The following properties hold for any integer $n \geq m$, given $a_m, \dots$ and $b_m,\dots$ sequences of real numbers.
      \begin{itemize}
        \item $\sum_{k=m }^{n} a_k + \sum_{k=m }^{n} b_k = \sum_{k=m }^{n} (a_k + b_k)$
        \item $c\cdot \sum_{k=m }^{n} a_k = \sum_{k=m }^{n} (c\cdot a_k)$, given some constant $c$
        \item $(\prod_{k=m}^{n}a_k) \cdot (\prod_{k=m}^{n}b_k) = \prod_{k=m}^{n} (a_k \cdot b_k)$
      \end{itemize}
    \end{theorem}
    \begin{theorem}
      The binomial theorem, also called $n$ choose $r$ is computed by using the following formula for $0 \le r \le n$:
      \begin{displaymath}
       \begin{pmatrix}
        n \\ r
       \end{pmatrix} 
       = \displaystyle\frac{n! }{r!(n-r)!}
      \end{displaymath}
    \end{theorem}
    
    
  \section{Mathmatical Induction} 
  Mathematical Induction is a two-step process, Take a statement in the form "For every integer $n \geq a $, a property P(n) holds true". We then apply the following:
  \begin{enumerate}
    \item The first step is called the \textbf{Basis Step}, this is where you show that the condition $P$ of your starting point $a$, is true $\rightarrow$ Show that $P(a)$ is true.
    \item The second step is called the \textbf{Inductive Step}, you show that for every integer $k \geq a$, if $P(k)$ is true, then $P(k+1)$ must also be true.
      \begin{enumerate}
        \item To perform this step, we must suppose that $P(k)$ holds, where $k \geq a$ (\textbf{Inductive Hypothesis}), therefore $P(k+1)$ must be true.
      \end{enumerate}
  \end{enumerate}
  To write an inductive proof formally, we must clearly state all steps and assumptions, take the following example:

  \textbf{Prove.} For every integer $n \geq 1, 1+2+\dots+n= \frac{n(n+1 )}{2}$ by \textbf{Induction}:

  \begin{proof}
    \underline{Basis Step} (\textit{Base Case}):
    \begin{displaymath}
      \textrm{Let } n = 1, P(1) \textrm{ holds because } \frac{1(1+1 )}{2} = \frac{1(2)}{2} = 1
    \end{displaymath}
    \underline{Inductive Step}:
    \begin{displaymath}
      \textrm{Let } n = k \textrm{ with } k \geq 1. \textrm{ Suppose that } P(k) \textrm{ is true (\textit{inductive hypothesis}). Thus we can show:}
    \end{displaymath}
    \begin{equation*}
      \begin{split}
        1+2+\dots+k+(k+1)&=(1+2+\dots+k)+(k+1)\\
                         &=\frac{k(k+1)}{2} +(k+1) \\
                         &=\frac{k(k+1 )}{2} +\frac{2(k+1 )}{2}\\
                         &=\frac{k^2 + k }{2} + \frac{2k+2 }{2} \\
                         &=\frac{k^2 + 3k +2 }{2}\\
                         &= \frac{(k+1)(k+2)}{2}
      \end{split}
    \end{equation*}
    Thus $1+2+\dots+(k+1) = \frac{(k+1)((k+1)+1)}{2}$ (notice how this is in the form $1+2+\dots+n=\frac{n(n+1 )}{2}$). Therefore we can conclude that $P(k+1)$ is true.
  \end{proof}
  \subsection{Recursion}
  Given a recursively defined sequence $a_i$, find an analytical formula using the following steps:
  \begin{enumerate}
    \item Find an explicit formula for $a_i$ by making educated guesses.
    \item Prove that the formula holds through induction.
  \end{enumerate}
  Take the following example:

  \textbf{Prove.} Given $b_0 = 1, b_n = \frac{b_n -1 }{1 - b_n -1}$ for $n>1$
  \begin{proof}
    Make an assumption for the formula of $b_n$, take the following guess:
    \begin{displaymath}
      b_n = \frac{1 }{n+1} \textrm{ for } n \geq 0 \textrm{ (\textit{Conjecture})}
    \end{displaymath}
    \underline{Basis Case}:
    \begin{displaymath}
      \textrm{Let } n = 0. P(0) \textrm{ is known to be true as } b_0 = 1 = \frac{1 }{0+1}
    \end{displaymath}
    \underline{Inductive Step:}
    \[
      \textrm{Let } n = k \textrm{ with } k \geq 0. \textrm{ Assume that $P(k)$ is true (\textit{inductive hypothesis}). Thus we can show:}
    \]
    \begin{displaymath}
      \begin{split}
        b_{k+1} &= \frac{b_k }{1+ b_k} \\
                &= \frac{\frac{1 }{k+1}}{1 + \frac{1 }{k+1}}\\
                &= \frac{\frac{1 }{k+1 }}{\frac{k+2 }{k+1}} \\
                &= \frac{1 }{k+2} \\
                &= \frac{1 }{(k+1)+1}
      \end{split}
    \end{displaymath}
    Hence $P(k+1)$ must be true.
  \end{proof}
  
  \subsection{Strong Induction}
  Strong induction follows the same steps as normal/weak induction, with the difference that the basis step may contain proofs for several values and $P(n)$ is assumed not just for a single $n$ but for all values $n$ through $k$, only then is the truth of $P(k+1)$ proved. The steps for strong induction are as follows:

  \textit{Let $P(n)$ be a property that is defined for integers $n$, let $a,b$ be fixed integers such that $a \le b$. Suppose the following statements:}
  \begin{enumerate}
    \item $P(a), P(a+1),\dots,P(b)$ are all true. (\textbf{Basis Step})
    \item For every integer $k \geq b$, if $P(i)$ is true for each integer $i$ from $a$ through $k$, then $P(k+1)$ is true. (\textbf{Inductive Step})
  \end{enumerate}
  \textit{then the statement "For every integer $n \geq a, P(n)$"} is true. The supposition that $P(i)$ is true is the inductive hypothesis in this case.
  
  \textbf{Prove.} Given a sequence $a_i$, 
  \[
    a_0 = 12, a_1 = 29 \quad a_n = 5a_{n-1} - 6a_{n-2}
  \]
  Show that for all $n \geq 0$,
  \[
    a_n = 5 \cdot 3^n + 7 \cdot 2^n
  \]
  \begin{proof}
    Proceed by induction on $n$.

    \underline{Basis Step:} We take that $P(0)$ and $P(1)$ are true.
    
    \underline{Inductive Step:} Let $n=k+1$. Assume $P(i)$ is true for $k \geq i \geq 0$ (\textit{inductive Hypothesis}). Thus:
    \begin{displaymath}
      \begin{split}
        a_k &= 5 \cdot 3^k + 7 \cdot 2^k \\
        a_{k-1} &= 5 \cdot 3^{k-1} + 7 \cdot 2^{k-1} \\
        a_{k+1} &= 5(5\cdot 3^k + 7\cdot 2^k) - 6(5\cdot 3^{k-1} + 7\cdot 2^{k-1})\\
                &= 25 \cdot 3^k + 35 \cdot 2^k - 30 \cdot 3^{k-1} - 42 \cdot 2^{k-1} \\ 
                &= 25 \cdot 3^k + 35 \cdot 2^k - 10 \cdot 3^k - 21 \cdot 2^k \\ 
                &= 15 \cdot 3^k + 14 \cdot 2^k \\
                &= 5 \cdot 3^{k+1} + 7 \cdot 2^{k+1}
      \end{split}
    \end{displaymath}
    Hence $P(k+1)$ is true.
  \end{proof}
  
  \section{Placeholder} 
  \section{Placeholder} 
  \section{Placeholder} 
  \section{Placeholder} 
  \section{Placeholder} 
  \section{Placeholder} 
\end{document}

