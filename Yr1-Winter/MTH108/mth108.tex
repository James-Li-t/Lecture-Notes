\documentclass[a4paper]{article}
\input{pre}

\title{MTH108 | Linear Algebra}
\author{James Li | 501022159 \and Professor: K. Q. Lan\and Email: klan@torontomu.ca}
\date{}
\renewcommand{\contentsname}{\centering Content by Week}
\begin{document}
  \maketitle
  \tableofcontents
  \newpage
  \section{Euclidean Spaces} 
  A Euclidean Space is a mathematical space in which points and lines can be represented by a set of coordinates in the respective dimension of the space, and every point can be represented in a defined set. For example:
  $$
  \R^3 = {(x,y,z); x,y,z \in \R}
  $$
  is three-dimensional space represented using coordinates in terms of $(x,y,z)$, where $x,y,z \in \R$.
  \begin{theorem}
    Given two vectors $\vec{a},\vec{b}$ and some constant $k$, $\vec{a}$ and $\vec{b}$ are called \textbf{parallel} if:
    \begin{displaymath}
      \vec{a} = k\vec{b} \Leftrightarrow \vec{a} // \vec{b}
    \end{displaymath}
    
  \end{theorem}
  
  \subsection{Products of Vectors with Constants}
  \begin{theorem}
    Given a constant $k$ in $\R$ and some vector $\vec{a}$ in $\R^2$ , the product of $k \vec{a}$ is:
    \begin{displaymath}
      k \vec{a} = k(x_1,y_1) = (k\cdot x_1, k\cdot y_1), x,y \in \R
    \end{displaymath}
  \end{theorem}
  To represent a vector in Linear Algebra, we can use the following notation (\textit{using the previously mentioned vector $\vec{a}$ as an example}):
  $$
  \vec{a} = (x_1,y_1) = 
  \begin{pmatrix}
   x_1 \\ y_1 
  \end{pmatrix}
  $$
  \subsubsection{Examples:}
  Take $\vec{a} = (\begin{smallmatrix} 2 \\ -1\end{smallmatrix})$ and $\vec{b} = (\begin{smallmatrix} 4 \\ -1\end{smallmatrix})$, compute $-2\vec{a} + 3\vec{b}$:
  \begin{equation}
    \label{eq:1}
    \begin{split}
      -2\vec{a} + 3\vec{b} &= 2\begin{pmatrix}
       2 \\ 1 
      \end{pmatrix} + 3 \begin{pmatrix}
       4 \\ -1 
      \end{pmatrix}\\
                           &= \begin{pmatrix}
                           -4 \\ 2 
                           \end{pmatrix} + \begin{pmatrix}
                           12 \\ -3 
                           \end{pmatrix} \\
                           &= \begin{pmatrix}
                           8 \\ -1 
                           \end{pmatrix}
    \end{split}
  \end{equation}
  \subsection{Products of two Vectors}
  \begin{theorem}
    Given two vectors in $\R^n ,\vec{a} = (\begin{smallmatrix}x_1 \\ \vdots \\ x_n\end{smallmatrix}), \vec{b} = (\begin{smallmatrix}y_1\\ \vdots \\ y_n\end{smallmatrix})$, the product of $\vec{a} \cdot \vec{b}$ is:
    \begin{displaymath}
      \begin{split}
        \vec{a} \cdot \vec{b} &= \begin{pmatrix}
          x_1 & \dots & x_n 
        \end{pmatrix} \cdot \begin{pmatrix}
          y_1 \\ \vdots \\ y_n
      \end{pmatrix}\\
                              &= x_1y_1 + x_2y_2 + \dots + x_ny_n
      \end{split}
    \end{displaymath}
   This is known as the \textbf{Dot Product}.
  \end{theorem}
  \subsubsection{Examples:}
  Take $\vec{A} = (\begin{smallmatrix} 1 \\ -1 \\ 2 \\ 3\end{smallmatrix})$ and $\vec{b} = (\begin{smallmatrix}2 \\ 1 \\ -1 \\ 1\end{smallmatrix})$, Find the dot product of $\vec{a} \cdot \vec{b}$:
  \begin{equation}
    \label{eq:2}
    \begin{split}
      \vec{a} \cdot \vec{b} &= (2) + (-1) + (-2) + (3) \\
                            &= 2
    \end{split}
  \end{equation}
  \subsection{Other Properties of Vector Products}
  Given some $\vec{a},\vec{b}$ in $\R^n$ and some constant $k$, the following properties apply:
  \begin{itemize}
    \item $(\vec{a}+\vec{b})(\vec{a}+\vec{b}) = \vec{a}^2+2\vec{a}\vec{b}+\vec{b}^2$
    \item $\vec{a} \cdot \vec{b} = \vec{b} \cdot \vec{a}$
    \item $\vec{a} \cdot (\vec{b}+\vec{c}) = \vec{a}\vec{b} + \vec{a}\vec{c}$
    \item $\vec{a}(k\vec{b}) = k\vec{a}\vec{b}$
    \item $\vec{a} \cdot \vec{a} = \vec{a}^2 = x_1^2 + x_2^2 + \dots + x_n^2$
  \end{itemize}
  The midpoint $C(z_1,z_2,\dots,z_n)$ of a line from $A(x_1,x_2,\dots,x_n)$ to $B(y_1,y_2,\dots,y_n)$ is calculated using the following forumla:
  \[
    \begin{pmatrix}
    z_1 \\ z_2 \\ \vdots \\ z_n
    \end{pmatrix}
    = (1-t)
    \begin{pmatrix}
     x_1 \\ x_2 \\ \vdots \\ x_n 
    \end{pmatrix}
    + t
    \begin{pmatrix}
     y_1 \\ y_2 \\ \vdots \\ y_n 
    \end{pmatrix}
    \quad
    \textrm{for t such that }
    0 \le t \le 1
  \]
  We can simplify this to calculate every $z_i$:
  \[
    z_i = \displaystyle\frac{x_i+y_i}{2}
  \]
  \subsubsection{Examples:}
  Given $A(1,-2)$ and $B(-3,4)$, find the midpoint $C(x,y)$:
  \begin{equation}
    \begin{split}
      x &= \displaystyle\frac{1+(-3 )}{2} \\
        &= -1\\
      y &= \displaystyle\frac{(-2)+4 }{2} \\
        &= 1
    \end{split}
  \end{equation}
  Therefore $C = (-1,1)$.
  \subsection{Norm and Angle}
  The magnitude of a vector in $\R^n$ is called the \textbf{Norm}, it is notated and defined as:
  \[
    ||\vec{a}|| = \sqrt{x_1^2 + x_2^2 + \dots + x_n^2} \quad (\vec{a}\textrm{ is some vector in }\R^n)
  \]
  We can use this definition to demonstrate some inequalities and properties (\textit{Given some $\vec{a},\vec{b} \in \R^n$ and some constant $k$}):
  \begin{itemize}
    \item $|| \vec{a} + \vec{b} ||  \le ||\vec{a}|| + ||\vec{b}||$ \textit{(Triangle Inequality)}
    \item $|| \vec{a} + \vec{b} || \geq ||\vec{a}|| - ||\vec{b}||$
    \item $|| k \vec{a} || = |k| \cdot || \vec{a}||$
    \item $|| \displaystyle\frac{\vec{a}}{||\vec{a}||}|| = 1$
    \item $\vec{a} \cdot \vec{b} = 0 \Leftrightarrow \vec{a} \bot \vec{b}$ \textit{(Orthagonal)}
    \item $\cos \theta = \displaystyle\frac{\vec{a}\cdot\vec{b}}{||\vec{a}||  ||\vec{b}||}$
  \end{itemize}
  \subsection{Determinants}
  The determinant of a matrix is a number that can be calculated using the following formula (\textit{for a 2x2 matrix}):
  \[
    \begin{vmatrix}
      a & b \\ c & d
    \end{vmatrix}
    = ad - bc
  \]
  Given $\vec{a},\vec{b} \in \R^n$, the \textbf{Grand Determinant} of $\vec{a}\vec{b}$ is defined as:
  \[
    G(\vec{a},\vec{b})
    = \begin{vmatrix}
      \vec{a} \cdot \vec{a} & \vec{a} \cdot \vec{b} \\
      \vec{b} \cdot \vec{a} & \vec{b} \cdot \vec{b}
    \end{vmatrix}
    = ||\vec{a}||^2 \ ||\vec{b}||^2 - (\vec{a}\vec{b})^2
  \]
  \begin{lemma}
    The \textbf{Cauchy Inequality} states:
    \[
      G (\vec{a},\vec{b}) \geq 0 \Rightarrow |\vec{a} \cdot \vec{b}| \le || \vec{a}|| \cdot || \vec{b} ||
    \]
  \end{lemma}

  \section{Placeholder} 
  \section{Placeholder} 
  \section{Placeholder} 
  \section{Placeholder} 
  \section{Placeholder} 
  \section{Placeholder} 
  \section{Placeholder} 
  \section{Placeholder} 
  \section{Placeholder} 
\end{document}
