\documentclass[a4paper]{article}
\input{pre}

\title{MTH108 | Linear Algebra}
\author{James Li | 501022159 \and Professor: K. Q. Lan\and Email: klan@torontomu.ca}
\date{}
\renewcommand{\contentsname}{\centering Content by Week}
\begin{document}
  \maketitle
  \tableofcontents
  \newpage
  \section{Euclidean Spaces} 
  A Euclidean Space is a mathematical space in which points and lines can be represented by a set of coordinates in the respective dimension of the space, and every point can be represented in a defined set. For example:
  $$
  \R^3 = {(x,y,z); x,y,z \in \R}
  $$
  is three-dimensional space represented using coordinates in terms of $(x,y,z)$, where $x,y,z \in \R$.
  \begin{theorem}
    Given two vectors $\vec{a},\vec{b}$ and some constant $k$, $\vec{a}$ and $\vec{b}$ are called \textbf{parallel} if:
    \begin{displaymath}
      \vec{a} = k\vec{b} \Leftrightarrow \vec{a} // \vec{b}
    \end{displaymath}
    
  \end{theorem}
  
  \subsection{Products of Vectors with Constants}
  \begin{theorem}
    Given a constant $k$ in $\R$ and some vector $\vec{a}$ in $\R^2$ , the product of $k \vec{a}$ is:
    \begin{displaymath}
      k \vec{a} = k(x_1,y_1) = (k\cdot x_1, k\cdot y_1), x,y \in \R
    \end{displaymath}
  \end{theorem}
  To represent a vector in Linear Algebra, we can use the following notation (\textit{using the previously mentioned vector $\vec{a}$ as an example}):
  $$
  \vec{a} = (x_1,y_1) = 
  \begin{pmatrix}
   x_1 \\ y_1 
  \end{pmatrix}
  $$
  \subsubsection{Examples:}
  Take $\vec{a} = (\begin{smallmatrix} 2 \\ -1\end{smallmatrix})$ and $\vec{b} = (\begin{smallmatrix} 4 \\ -1\end{smallmatrix})$, compute $-2\vec{a} + 3\vec{b}$:
  \begin{equation}
    \label{eq:1}
    \begin{split}
      -2\vec{a} + 3\vec{b} &= 2\begin{pmatrix}
       2 \\ 1 
      \end{pmatrix} + 3 \begin{pmatrix}
       4 \\ -1 
      \end{pmatrix}\\
                           &= \begin{pmatrix}
                           -4 \\ 2 
                           \end{pmatrix} + \begin{pmatrix}
                           12 \\ -3 
                           \end{pmatrix} \\
                           &= \begin{pmatrix}
                           8 \\ -1 
                           \end{pmatrix}
    \end{split}
  \end{equation}
  \subsection{Products of two Vectors}
  \begin{theorem}
    Given two vectors in $\R^n ,\vec{a} = (\begin{smallmatrix}x_1 \\ \vdots \\ x_n\end{smallmatrix}), \vec{b} = (\begin{smallmatrix}y_1\\ \vdots \\ y_n\end{smallmatrix})$, the product of $\vec{a} \cdot \vec{b}$ is:
    \begin{displaymath}
      \begin{split}
        \vec{a} \cdot \vec{b} &= (x1,\dots, x_n) \cdot \begin{pmatrix}
          y_1 \\ \vdots \\ y_n
      \end{pmatrix}\\
                              &= x_1y_1 + x_2y_2 + \dots + x_ny_n
      \end{split}
    \end{displaymath}
   This is known as the \textbf{Dot Product}.
  \end{theorem}
  \subsubsection{Examples:}
  Take $\vec{A} = (\begin{smallmatrix} 1 \\ -1 \\ 2 \\ 3\end{smallmatrix})$ and $\vec{b} = (\begin{smallmatrix}2 \\ 1 \\ -1 \\ 1\end{smallmatrix})$, Find the dot product of $\vec{a} \cdot \vec{b}$:
  \begin{equation}
    \label{eq:2}
    \begin{split}
      \vec{a} \cdot \vec{b} &= (2) + (-1) + (-2) + (3) \\
                            &= 2
    \end{split}
  \end{equation}
  
  
  \section{Placeholder} 
  \section{Placeholder} 
  \section{Placeholder} 
  \section{Placeholder} 
  \section{Placeholder} 
  \section{Placeholder} 
  \section{Placeholder} 
  \section{Placeholder} 
  \section{Placeholder} 
\end{document}
