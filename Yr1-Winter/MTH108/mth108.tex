\documentclass[a4paper]{article}
\input{pre}

\title{MTH108 | Linear Algebra}
\author{James Li | 501022159 \and Professor: K. Q. Lan\and Email: klan@torontomu.ca}
\date{}
\renewcommand{\contentsname}{\centering Content by Week}
\begin{document}
  \maketitle
  \tableofcontents
  \newpage
  \section{Euclidean Spaces} 
  A Euclidean Space is a mathematical space in which points and lines can be represented by a set of coordinates in the respective dimension of the space, and every point can be represented in a defined set. For example:
  $$
  \R^3 = {(x,y,z); x,y,z \in \R}
  $$
  is three-dimensional space represented using coordinates in terms of $(x,y,z)$, where $x,y,z \in \R$.
  \begin{theorem}
    Given two vectors $\vec{a},\vec{b}$ and some constant $k$, $\vec{a}$ and $\vec{b}$ are called \textbf{parallel} if:
    \begin{displaymath}
      \vec{a} = k\vec{b} \Leftrightarrow \vec{a} // \vec{b}
    \end{displaymath}
    
  \end{theorem}
  
  \subsection{Products of Vectors with Constants}
  \begin{theorem}
    Given a constant $k$ in $\R$ and some vector $\vec{a}$ in $\R^2$ , the product of $k \vec{a}$ is:
    \begin{displaymath}
      k \vec{a} = k(x_1,y_1) = (k\cdot x_1, k\cdot y_1), x,y \in \R
    \end{displaymath}
  \end{theorem}
  To represent a vector in Linear Algebra, we can use the following notation (\textit{using the previously mentioned vector $\vec{a}$ as an example}):
  $$
  \vec{a} = (x_1,y_1) = 
  \begin{pmatrix}
   x_1 \\ y_1 
  \end{pmatrix}
  $$
  \subsubsection{Examples:}
  Take $\vec{a} = (\begin{smallmatrix} 2 \\ -1\end{smallmatrix})$ and $\vec{b} = (\begin{smallmatrix} 4 \\ -1\end{smallmatrix})$, compute $-2\vec{a} + 3\vec{b}$:
  \begin{equation}
    \label{eq:1}
    \begin{split}
      -2\vec{a} + 3\vec{b} &= 2\begin{pmatrix}
       2 \\ 1 
      \end{pmatrix} + 3 \begin{pmatrix}
       4 \\ -1 
      \end{pmatrix}\\
                           &= \begin{pmatrix}
                           -4 \\ 2 
                           \end{pmatrix} + \begin{pmatrix}
                           12 \\ -3 
                           \end{pmatrix} \\
                           &= \begin{pmatrix}
                           8 \\ -1 
                           \end{pmatrix}
    \end{split}
  \end{equation}
  \subsection{Products of two Vectors}
  \begin{theorem}
    Given two vectors in $\R^n ,\vec{a} = (\begin{smallmatrix}x_1 \\ \vdots \\ x_n\end{smallmatrix}), \vec{b} = (\begin{smallmatrix}y_1\\ \vdots \\ y_n\end{smallmatrix})$, the product of $\vec{a} \cdot \vec{b}$ is:
    \begin{displaymath}
      \begin{split}
        \vec{a} \cdot \vec{b} &= \begin{pmatrix}
          x_1 & \dots & x_n 
        \end{pmatrix} \cdot \begin{pmatrix}
          y_1 \\ \vdots \\ y_n
      \end{pmatrix}\\
                              &= x_1y_1 + x_2y_2 + \dots + x_ny_n
      \end{split}
    \end{displaymath}
   This is known as the \textbf{Dot Product}.
  \end{theorem}
  \subsubsection{Examples:}
  Take $\vec{A} = (\begin{smallmatrix} 1 \\ -1 \\ 2 \\ 3\end{smallmatrix})$ and $\vec{b} = (\begin{smallmatrix}2 \\ 1 \\ -1 \\ 1\end{smallmatrix})$, Find the dot product of $\vec{a} \cdot \vec{b}$:
  \begin{equation}
    \label{eq:2}
    \begin{split}
      \vec{a} \cdot \vec{b} &= (2) + (-1) + (-2) + (3) \\
                            &= 2
    \end{split}
  \end{equation}
  \subsection{Other Properties of Vector Products}
  Given some $\vec{a},\vec{b}$ in $\R^n$ and some constant $k$, the following properties apply:
  \begin{itemize}
    \item $(\vec{a}+\vec{b})(\vec{a}+\vec{b}) = \vec{a}^2+2\vec{a}\vec{b}+\vec{b}^2$
    \item $\vec{a} \cdot \vec{b} = \vec{b} \cdot \vec{a}$
    \item $\vec{a} \cdot (\vec{b}+\vec{c}) = \vec{a}\vec{b} + \vec{a}\vec{c}$
    \item $\vec{a}(k\vec{b}) = k\vec{a}\vec{b}$
    \item $\vec{a} \cdot \vec{a} = \vec{a}^2 = x_1^2 + x_2^2 + \dots + x_n^2$
  \end{itemize}
  The midpoint $C(z_1,z_2,\dots,z_n)$ of a line from $A(x_1,x_2,\dots,x_n)$ to $B(y_1,y_2,\dots,y_n)$ is calculated using the following forumla:
  \[
    \begin{pmatrix}
    z_1 \\ z_2 \\ \vdots \\ z_n
    \end{pmatrix}
    = (1-t)
    \begin{pmatrix}
     x_1 \\ x_2 \\ \vdots \\ x_n 
    \end{pmatrix}
    + t
    \begin{pmatrix}
     y_1 \\ y_2 \\ \vdots \\ y_n 
    \end{pmatrix}
    \quad
    \textrm{for t such that }
    0 \le t \le 1
  \]
  We can simplify this to calculate every $z_i$:
  \[
    z_i = \displaystyle\frac{x_i+y_i}{2}
  \]
  \subsubsection{Examples:}
  Given $A(1,-2)$ and $B(-3,4)$, find the midpoint $C(x,y)$:
  \begin{equation}
    \begin{split}
      x &= \displaystyle\frac{1+(-3 )}{2} \\
        &= -1\\
      y &= \displaystyle\frac{(-2)+4 }{2} \\
        &= 1
    \end{split}
  \end{equation}
  Therefore $C = (-1,1)$.
  \subsection{Norm and Angle}
  The magnitude of a vector in $\R^n$ is called the \textbf{Norm}, it is notated and defined as:
  \[
    ||\vec{a}|| = \sqrt{x_1^2 + x_2^2 + \dots + x_n^2} \quad (\vec{a}\textrm{ is some vector in }\R^n)
  \]
  We can use this definition to demonstrate some inequalities and properties (\textit{Given some $\vec{a},\vec{b} \in \R^n$ and some constant $k$}):
  \begin{itemize}
    \item $|| \vec{a} + \vec{b} ||  \le ||\vec{a}|| + ||\vec{b}||$ \textit{(Triangle Inequality)}
    \item $|| \vec{a} + \vec{b} || \geq ||\vec{a}|| - ||\vec{b}||$
    \item $|| k \vec{a} || = |k| \cdot || \vec{a}||$
    \item $|| \displaystyle\frac{\vec{a}}{||\vec{a}||}|| = 1$
    \item $\vec{a} \cdot \vec{b} = 0 \Leftrightarrow \vec{a} \bot \vec{b}$ \textit{(Orthagonal)}
    \item $\cos \theta = \displaystyle\frac{\vec{a}\cdot\vec{b}}{||\vec{a}||  ||\vec{b}||}$
  \end{itemize}
  \subsection{Determinants}
  The determinant of a matrix is a number that can be calculated using the following formula (\textit{in $\R^2$}):
  \[
    \begin{vmatrix}
      a & b \\ c & d
    \end{vmatrix}
    = ad - bc
  \]
  Given $\vec{a},\vec{b} \in \R^n$, the \textbf{Gram Determinant} of $\vec{a}\vec{b}$ is defined as:
  \[
    G(\vec{a},\vec{b})
    = \begin{vmatrix}
      \vec{a} \cdot \vec{a} & \vec{a} \cdot \vec{b} \\
      \vec{b} \cdot \vec{a} & \vec{b} \cdot \vec{b}
    \end{vmatrix}
    = ||\vec{a}||^2 \ ||\vec{b}||^2 - (\vec{a}\vec{b})^2
  \]
  Given some vectors $\vec{a},\vec{b} \in \R^3$, the Gram Determinant can be calculated using the following formula:
  \[
    G(\vec{a},\vec{b}) = \begin{vmatrix}
      x_1 & x_2 \\ y_1 & y_2 
    \end{vmatrix}^2 +
    \begin{vmatrix}
      x_1 & x_3 \\ y_1 & y_3
    \end{vmatrix}^2 +
    \begin{vmatrix}
      x_2 & x_3 \\ y_2 & y_3
    \end{vmatrix}^2
  \]
  \begin{lemma}
    The \textbf{Cauchy Inequality} states:
    \[
      G (\vec{a},\vec{b}) \geq 0 \Rightarrow |\vec{a} \cdot \vec{b}| \le || \vec{a}|| \cdot || \vec{b} ||
    \]
  \end{lemma}

  \section{Projections, Linear Combinations and Span} 
  For some $\vec{a},\vec{b} \in \R^n$, a projection of $\vec{a}$ on $\vec{b}$ is defined as:
  \[
    \vectorproj[\vec{a}]{\vec{b}} = \frac{\vec{a}\cdot\vec{b}}{||\vec{a}||^2} \cdot \vec{a}
  \]
  The norm of a projection is defined as:
  \[
    ||\vectorproj[\vec{a}]{\vec{b}}|| = \frac{\vec{a}\cdot\vec{b}}{||\vec{a}||^2}\cdot ||\vec{a}||
  \]
  \subsection{Projection and Area of a parallelogram}
  \begin{figure}[h]
    \begin{center}
      \includegraphics[width=0.5\textwidth]{img/2023-01-23-20-16-23.png}
    \end{center}
  \end{figure}
  
  Given the above figure, we can determine formulas for calculating properties of the parallelogram:

  \begin{enumerate}
    \item $h = ||\vec{BC}|| = ||\vec{b} - \vectorproj[\vec{a}]{\vec{b}}||$
    \item $Area = ||\vec{a}|| \cdot ||\vec{b} - \vectorproj[\vec{a}]{\vec{b}}||$
    \item $\sin(\theta) = \frac{||\vec{b}-\vectorproj[\vec{a}]{\vec{b}}||}{||\vec{b}||}$
    \item $\cos(\theta) = \frac{||\vectorproj[\vec{a}]{\vec{b}}||}{||\vec{b}||} = \frac{\vec{a}\cdot\vec{b}}{||\vec{a}||||\vec{b}||}$
  \end{enumerate}
  The generalized formula for calculating the area of some $\vec{a},\vec{b} \in \R^n$ is as follows:
  \[
    A(\vec{a},\vec{b}) = \sqrt{G(\vec{a},\vec{b})}
  \]
  \subsection{Linear Combinations and Span}
  Take some vectors $\vec{a},\vec{b} \in \R^n$, the linear combination($A$) of these vectors is simply:
  \[
    m \vec{a} + n\vec{b} = A\qquad n,m \in \R
  \]
  Span can be thought of as the set of values that you can reach by changing the constant in a linear combination of vectors, taking $\vec{a} = (1,2)$ and $\vec{b} = (2,3)$ the span would then be:
  \[
    span(a,b) = \R^2
  \]
  This is because you could theoretically reach any point in $\R^2$ by changing values of $m,n \in \R$, the exception to this is when both vectors are zero-vectors or when both vectors "align", in that case the set of values possible would only be the values extending the single line \textit{(e.g $\vec{a} = (1,0),\vec{b}=(-1,0)$) }.

  \section{Matrices} 
  A \textbf{matrix} $A$,  of $n \times m$ is defined as a rectangular array of $mn$ numbers arranged in $m,n$ rows and columns:
  \[
    A = \begin{pmatrix}
      a_{1,1} & a_{1,2} & a_{1,3}& \dots & a_{1,n} \\
      a_{2,1} & a_{2,2} & a_{2,3}&\dots & a_{2,n} \\
      a_{3,1} & a_{3,2} & a_{3,3}&\dots & a_{3,n} \\
      \vdots & \vdots & \vdots&\ddots & \vdots\\
      a_{m,1} & a_{m,2} & a_{m,3}&\dots & a_{m,n} \\
    \end{pmatrix}
  \]
  If we take only either columns or rows we can define:
  \[
    a_1 = \begin{pmatrix}
      a_{1,1}  & a_{1,2} & \dots & a_{1,n}
    \end{pmatrix} \textrm{ This is called a \textbf{Row Matrix}}
  \]
  \[
    a_1 = \begin{pmatrix}
      a_{1,1}  \\ a_{2,1} \\ \dots \\ a_{m,1}
    \end{pmatrix} \textrm{ This is called a \textbf{Column Matrix}}
  \]
  \subsection{Properties of Matrices}
  The \textbf{size} of a matrix is simply its number of rows $\times$ its number of columns:
  \[
    size\begin{pmatrix}
      1 & 2 & 3 \\ 4 & 5 & 6 
    \end{pmatrix} = 2\times 3 = 6
  \]
  The \textbf{transpose} of a matrix is defined as the matrix itself but with the rows and columns swapped, it is denoted with $^T$, take the previously defined matrix $A$, the transpose of $A = A^T$, would be denoted as:
  \[
    A^T = \begin{pmatrix}
      a_{1,1} & a_{2,1} & a_{3,1}& \dots & a_{m,1} \\
      a_{1,2} & a_{2,2} & a_{3,2}&\dots & a_{m,2} \\
      a_{1,3} & a_{2,3} & a_{3,3}&\dots & a_{m,3} \\
      \vdots & \vdots & \vdots&\ddots & \vdots\\
      a_{1,n} & a_{2,n} & a_{3,n}&\dots & a_{m,n} \\
    \end{pmatrix}
  \]
  Two matrices $A,B$ are said to be \textbf{equal}, if their size are the same and the respective entries are equal:
  \[
    A = \begin{pmatrix}
      1 & 2 & 3 \\ 4 & 5 & 6 \\ 7 & 8 & 9 
    \end{pmatrix} =
    \begin{pmatrix}
      1 & 2 & 3 \\ 4 & 5 & 6 \\ 7 & 8 & 9 
    \end{pmatrix} = B
  \]
  Matrix \textbf{addition, subtraction and scalar multiplication} can all be achieved in the same method as their vector counterparts.
  \subsection{Product of a Vector and Matrix}
  The product of matrix $A$ with a row/column vector $X$ is a vector where the values are each entry in the row/column of $A$ multiplied by the $n$th element in $X$ for that row/column, then summed. For example, take $X = (x_1,x_2,x_3,\dots,x_n)$
  \[
    AX = \begin{pmatrix}
      a_{1,1}x_{1} &+& a_{1,2}x_{1} &+& a_{1,3}x_{1} &+& \dots &+& a_{1,n}x_{1}\\
      a_{2,1}x_{2} &+& a_{2,2}x_{2} &+& a_{2,3}x_{2} &+& \dots &+& a_{1,n}x_{2}\\
      a_{3,1}x_{3} &+& a_{3,2}x_{3} &+& a_{3,3}x_{3} &+& \dots &+& a_{1,n}x_{3}\\
      \vdots & \ & \vdots & \ & \vdots & \ & \vdots & \ & \vdots \\
      a_{m,1}x_{n} &+& a_{m,2}x_{n} &+& a_{m,3}x_{n} &+& \dots &+& a_{m,n}x_{n}\\
    \end{pmatrix}
  \]
  If $X = (\begin{smallmatrix} x_1 \\ x_2 \\ \dots \\ x_n  \end{smallmatrix})$ instead, then $AX$ would be:
  \[
    AX = \begin{pmatrix}
      a_{1,1}x_{1} &+& a_{1,2}x_{2} &+& a_{1,3}x_{3} &+& \dots &+& a_{1,n}x_{n}\\
      a_{2,1}x_{1} &+& a_{2,2}x_{2} &+& a_{2,3}x_{3} &+& \dots &+& a_{1,n}x_{n}\\
      a_{3,1}x_{1} &+& a_{3,2}x_{2} &+& a_{3,3}x_{3} &+& \dots &+& a_{1,n}x_{n}\\
      \vdots & \ & \vdots & \ & \vdots & \ & \vdots & \ & \vdots \\
      a_{m,1}x_{1} &+& a_{m,2}x_{2} &+& a_{m,3}x_{3} &+& \dots &+& a_{m,n}x_{n}\\
    \end{pmatrix}
  \]
  Note that a matrix and vector can only be multiplied if the number of columns of the matrix is equal to the number of elements in the vector. This can also be called the \textbf{image} of $X$ under $A$.
  \subsection{Product of two Matrices}
  Given matrcies $A,B$, their product $AB$, can be calculated by taking the \textbf{dot product} between the first row of $A$ with the first column of $B$, then the first row of $A$ with the second column of $B$, until the $n$th column of $B$, then you repeat the process until you reach the $n$th row of $A$. Take the following example:
  \[
    A = \begin{pmatrix}
      1 & 2 \\ 3 & 4 
    \end{pmatrix}
    B = \begin{pmatrix}
      5 & 6 \\ 7 & 8 
    \end{pmatrix}
  \]
  \[
    \begin{split}
      AB = &((1,2) \cdot (5,7) + (1,2) \cdot (6,8))\\
         & ((3,4)\cdot (5,7) + (3,4) \cdot (6,8))\\
         &= \begin{pmatrix}
           19 & 22 \\ 43 & 50
         \end{pmatrix}
    \end{split}
  \]
  Two matrices $A,B$ with sizes $size(A) = n \times m$ and $size(B) = i \times j$, with $n,m,i,j \in \R$, $AB$ can only be evaluated if $m = i$. The size of the resulting matrix $AB$ will be $n\times j$. Keep in mind that commutativity does apply in regard to matrix multiplication, hence $AB \not = BA$ may be true. Also keep in mind the following formula:

  \[
    (AB)^T = B^T A^T
  \]

  \subsection{Square Matrices}
  A square matrix is a matrix of $n \times n$ dimensions, for example:
  \[
    A = \begin{pmatrix}
      a_{1,1} & a_{1,2} & a_{1,3}& \dots & a_{1,n} \\
      a_{2,1} & a_{2,2} & a_{2,3}&\dots & a_{2,n} \\
      a_{3,1} & a_{3,2} & a_{3,3}&\dots & a_{3,n} \\
      \vdots & \vdots & \vdots&\ddots & \vdots\\
      a_{n,1} & a_{n,2} & a_{n,3}&\dots & a_{n,n} \\
    \end{pmatrix}
  \]
  The \textbf{trace} of a square matrix is defined as the sum of all the diagonal elements:
  \[
    tr(A) = a_{1,1} + a_{2,2} + \dots + a{n,n}
  \]
  If all the elemnts above/below the diagonal line in a matrix is 0, then the matrix is called a \textbf{lower/upper} triangle matrix respectively. If all values on either side of the diagonal are 0, then it is called a \textbf{upper and lower} triangle matrix.
  \newpage
  Let $A$ be a $n\times n $ matrix, the following properties then apply:
  \begin{itemize}
    \item If $A^T = A$ then $A$ is called \textbf{symmetric}.
    \item $A^n = \Pi^n_0 A$ 
    \item $A^0 = I$ where $I$ is an identity matrix with the same dimensions as $A$.
  \end{itemize}
  We define a diagonal matrix in the form $diag(a_{1,1},a_{2,2},\dots,a_{n,n})$ as follows:
  \[
    A = \begin{pmatrix}
      a_{1,1} & 0 & 0& \dots & 0 \\
      0 & a_{2,2} & 0&\dots & 0 \\
      0 & 0 & a_{3,3}&\dots & 0 \\
      \vdots & \vdots & \vdots&\ddots & \vdots\\
      0 & 0 & 0&\dots & a_{n,n} \\
    \end{pmatrix}
  \]
  Given some identity matrix $I$ with the same dimensions as the previous diagonal matrix, then the following property holds:
  \[
    IA = AI
  \]
  For a diagonal matrix, the following property also applies for some $k\in \R$:
  \[
    A = \begin{pmatrix}
      a_{1,1} & 0 & 0& \dots & 0 \\
      0 & a_{2,2} & 0&\dots & 0 \\
      0 & 0 & a_{3,3}&\dots & 0 \\
      \vdots & \vdots & \vdots&\ddots & \vdots\\
      0 & 0 & 0&\dots & a_{n,n} \\
    \end{pmatrix}^k = 
    A = \begin{pmatrix}
      a_{1,1}^k & 0 & 0& \dots & 0 \\
      0 & a_{2,2}^k & 0&\dots & 0 \\
      0 & 0 & a_{3,3}^k&\dots & 0 \\
      \vdots & \vdots & \vdots&\ddots & \vdots\\
      0 & 0 & 0&\dots & a_{n,n}^k \\
    \end{pmatrix}
  \]
  Keep in mind this \textbf{only} applies to diagonal matrices.
  \section{Row Echleon Form \& Reduced Row Echleon Form} 
  \subsection{Elementary Row Operations}
  There are 3 operations which can be performed on any matrix, with the result being considered \textbf{row equivalent} to the original matrix, meaning that they represent the same matrix (\textit{Think about how $\frac{4 }{3} = \frac{8 }{6}$}). From now on, the $n$th row of a matrix will be referred to as $Rn$, the operations are as follows:
  \begin{enumerate}
    \item \underline{Interchange two rows}\newline
      \textbf{Example:} Interchange $R1$ with $R2$ in the following matrix.
      \[
        \begin{pmatrix}
          1 & 2 & 3 & 4 \\  
          5 & 6 & 7 & 8 \\  
          9 & 10 & 11 & 12 \\  
          13 & 14 & 15 & 16 \\  
        \end{pmatrix} \overset{R1 \leftrightarrow R2}{\Rightarrow}
        \begin{pmatrix}
          5 & 6 & 7 & 8 \\  
          1 & 2 & 3 & 4 \\  
          9 & 10 & 11 & 12 \\  
          13 & 14 & 15 & 16 \\  
        \end{pmatrix}
      \]
    \item \underline{Multiply all elements in a row by some constant $>$ 0}\newline
      \textbf{Example:} Multiply $R1$ by 3.
      \[
        \begin{pmatrix}
          1 & 2 & 3 & 4 \\  
          5 & 6 & 7 & 8 \\  
          9 & 10 & 11 & 12 \\  
          13 & 14 & 15 & 16 \\  
        \end{pmatrix} \overset{R1 = 3R1}{\Rightarrow}
        \begin{pmatrix}
          3 & 6 & 9 & 12 \\  
          5 & 6 & 7 & 8 \\  
          9 & 10 & 11 & 12 \\  
          13 & 14 & 15 & 16 \\  
        \end{pmatrix}
      \]
    \item \underline{Add or Subtract some nonzero multiple of one row to another}\newline
      \textbf{Example:} Add $3R2$ to $R1$.
      \[
        \begin{pmatrix}
          1 & 2 & 3 & 4 \\  
          5 & 6 & 7 & 8 \\  
          9 & 10 & 11 & 12 \\  
          13 & 14 & 15 & 16 \\  
        \end{pmatrix} \overset{R1 = R1 + 3R2}{\Rightarrow}
        \begin{pmatrix}
          16 & 20 & 24 & 28 \\  
          5 & 6 & 7 & 8 \\  
          9 & 10 & 11 & 12 \\  
          13 & 14 & 15 & 16 \\  
        \end{pmatrix}
      \]
  \end{enumerate}
  \subsection{Row Echleon Form (REF)}
  For any matrix to be in row echleon form (REF), it must satisfy the following conditions:
  \begin{itemize}
    \item Every row of nonzero elements must be \underline{above} rows of all zero.
    \item Each leading nonzero element of a row must be in a column to the \underline{right} of the leading element in the row above it.
    \item Every element in the column \underline{below} a leading element must be zero.
  \end{itemize}
  Note however, every element in the column \underline{above} a leading element does not have to be zero. The following is a matrix in REF:
  \[
    \begin{pmatrix}
      \fbox{1} & 2 & 0 & 1 \\  
      0 & \fbox{2} & 3 & 1 \\  
      0 & 0 & \fbox{4} & 0 \\  
      0 & 0 & 0 & 0 \\  
    \end{pmatrix}
  \]
  Where the elements encased with \fbox{\textit{Element}} are the leading elements of that row.
  \subsection{Reduced Row Echleon Form (RREF)}
  For any matrix to be in reduced row echleon form (RREF), it must satisfy the following conditions:
  \begin{itemize}
    \item The matrix is already in REF.
    \item The leading nonzero entry in each row \textbf{must} be equal to 1.
    \item Each leading 1 is the \textbf{only} nonzero element in the entire column.
  \end{itemize}
  If we performed row operations on the matrix from REF definition, it would look as follows in RREF:
  \[
    \begin{pmatrix}
      1 & 0 & 0 & 0 \\  
      0 & 1 & 0 & \frac{1 }{2} \\  
      0 & 0 & 1 & 0 \\  
      0 & 0 & 0 & 0 \\  
    \end{pmatrix}
  \]
  \section{Placeholder} 
  \section{Placeholder} 
  \section{Placeholder} 
  \section{Placeholder} 
  \section{Placeholder} 
  \section{Placeholder} 
\end{document}
