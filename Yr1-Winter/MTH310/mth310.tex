\documentclass[a4paper]{article}
\input{pre}

\title{MTH310 Calculus \& Computational Methods II}
\author{James Li | 501022159 \and Professor: L. Kolasa\and Email: lkolasa@torontomu.ca}
\date{}

\renewcommand{\contentsname}{\centering Content by Week}

\begin{document}
  \maketitle
  \tableofcontents
  \newpage
  \section{Integration Practice} 
  \subsection{FToC I} 
  Recall that we can solve a definite integral using the following definition:
  $$
    \displaystyle\int_{a }^{b} f(x)dx = F(b) - F(a)
  $$ 
  \subsection{Change of Variables / U-Substitution}
  Suppose we need take the antiderivative of $\int 2x \cos(x^2)dx$, let us suppose that $g(x) = x^2$, then we know $g^\prime (x) = 2x$. We also know that $\int \cos (x) dx = \sin(x) + C$. If we combine these, we can derive the answer as:
  $$
  \sin (x^2) + C = \int 2x \cos(x^2)dx
  $$
  \begin{theorem}
    Let us take $u = g^\prime (x) \rightarrow \frac{du }{dx } = g^\prime (x)$ and $du = g^\prime (x) dx$. We can then derive the following:
      \begin{equation}
        \label{Usub}
        \begin{split}
          f(g(x)) &= \int (f \cdot g)^\prime x \\
                  &= \int f^\prime (g(x)) g^\prime(x) dx \\
                  &= \int f^\prime (u) du
        \end{split}
      \end{equation}
  \end{theorem}
  \subsubsection{Examples:}
  Consider the following substitution $u = 3x \rightarrow du = 3x \rightarrow \frac{1}{3} du = dx$, we can then solve:
  \begin{equation}
    \label{Example 1}
    \begin{split}
      \int \cos (3x) dx   
                        &= \frac{1}{3} \int \cos (u) du \\
                        &= \frac{1}{3} \sin (u) + C \\
                        &= \frac{1}{3} \sin (3x) + C
    \end{split}
  \end{equation}
  Consider the following substitution $u = 2x^2 +1 \rightarrow du = 4x\ dx $, we can then solve:
  \begin{equation}
    \label{Example 2}
    \begin{split}
      \int \frac{x }{2x^2+1} dx &= \frac{1}{4} \int \frac{du }{u}\\
                                &= \frac{1}{4} ln(u) + C\\
                                &= \frac{1}{4} ln (2x^2+1) + C
    \end{split}
  \end{equation}
  Consider the following substitution $ u = 1+x \rightarrow du = dx \rightarrow u - 1 = x$
  \begin{equation}
    \label{Example 3}
    \begin{split}
      \int x \sqrt{1+x}\ dx &= \int (u-1) u ^{\frac{1 }{2}} \\
                            &= \int u^{\frac{3 }{2}} - u^{\frac{1 }{2 }}du \\
                            &= \frac{2 }{5} u^{\frac{5 }{2 }} - \frac{2 }{3} u^{\frac{3 }{2}} + C\\
                            &= \frac{2 }{5} (1+x)^{\frac{5 }{2 }} - \frac{2 }{3} (1+x)^{\frac{3 }{2}} + C\\
    \end{split}
  \end{equation}
  \subsection{Area between two curves}
  \textbf{Recall:}

  Suppose $f(x) \geq 0$ is the area beneath the curve $0\le y \le f(x)$, where $a\le x \le b$, then:
  $$
    \displaystyle\int_{a}^{b} f(x) dx  
  $$
  is the area of the curve between $a,b$.

  \begin{theorem}
    Given functions $f(x), g(x)$ the area between the curves from $(a,b)$ can be represented as:
    \begin{displaymath}
     \displaystyle\int_{a }^{b}  (f(x)-g(x))dx
    \end{displaymath}
  \end{theorem}
  \subsubsection{Examples:}
  Given two functions $f(x) = 3x^2 + 12$ and $g(x) = 4x+4$, find the area between the curves from $(-3,3)$.
  \begin{equation}
    \label{Example1}
    \begin{split}
      \displaystyle\int_{-3 }^{3 } (f(x)-g(x))dx &= \displaystyle\int_{-3 }^{3} (3x^2+12) - (4x+4) dx\\
                                               &= \displaystyle\int_{-3 }^{3} 3x^2 - 4x + 8 dx \\
                                               &= x^3-2x^2+8x \|^{3}_{-3} \\
                                               &\vdots\\
                                               &= 102
    \end{split}
  \end{equation}
  Given two functions $(x^2+2)$ and $(2x+5)$, find the \textbf{enclosed} area between these two curves.

 \textit{Find $a,b$, where the lines intesect $\rightarrow 2x+5 = x^2 + 2 \dots (-1 \le x \le 3)$ then solve the integral.} 

  \begin{equation}
    \label{eq:3}
    \begin{split}
      \displaystyle\int_{-1}^{3} (f(x)-g(x)) dx &= \displaystyle\int_{-1 }^{3 } (x^2 + 2)-(2x+5)\\
                                                &\vdots \\
                                                &= \frac{32 }{3}
    \end{split}
  \end{equation}
  Given two functions $\sin(x)$ and $\cos(x)$ find the area between these two curves given that $0\le x \le \frac{\pi }{2}$.
  
  \textit{Find the intersection between the functions on the range given, then build the integral. This is easily done by observing the functions \textbf{geometrically.}}
  \begin{equation}
    \begin{split}
      \displaystyle\int_{0 }^{\frac{\pi }{4}} (f(x)-g(x)) dx &= \displaystyle\int_{0 }^{\frac{\pi }{4}}\cos(x)-\sin(x)dx+\displaystyle\int_{\frac{\pi }{4}}^{\frac{\pi }{2}}\sin(x)-\cos(x)  dx \\ 
                                                             &\vdots\\
                                                             &= 2(\sqrt{2}-1)
    \end{split}
  \end{equation}
  \newpage
  \section{Integration in Theory/Application} 
  \subsection{Volume of Revolution}
  Visualize a cyclindrical shape from $(a,b)$ built by rotating some curve defined by $f(x)$ about some $x$-axis
  \begin{figure}[h]
    \begin{center}
      \includegraphics[width=0.5\textwidth]{img/2023-01-23-20-57-11.png}
    \end{center}
    \caption{\textit{Cylinder built by rotating $f(x)$ about the $x$-axis}}
  \end{figure}

  We can define the volume of this cylinder as the cross section $\times$ thickness, similar to calculating the integral in a 2-dimensional plane. The integral for the volume of this cyclinder is then written as:
  \[
    V = \pi r^2 h 
  \]
  \[
    V =\int_{a }^{b} \pi f(x)^2 dx
  \]

  Assuming another function $g(x)$ also rotated about the same axis, creating a hollow inside of the cylinder, we would then need to subtract the hollow section from the full volume:
  \[
    V = \pi (\underset{outer}{r_o} - \underset{inner}{r_i})^2h
  \]
  \[
    V = \int_{a }^{b} \pi (\underset{outer}{f(x)} - \underset{inner}{g(x)})^2 dx
  \]
  
  \section{Placeholder} 
  \section{Placeholder} 
  \section{Placeholder} 
  \section{Placeholder} 
  \section{Placeholder} 
  \section{Placeholder} 
  \section{Placeholder} 
  \section{Placeholder} 
\end{document}

