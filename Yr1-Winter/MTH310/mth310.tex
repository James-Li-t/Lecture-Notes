\documentclass[a4paper]{article}
\input{pre}


\title{MTH310 Calculus \& Computational Methods II}
\author{James Li | 501022159 \and Professor: L. Kolasa\and Email: lkolasa@torontomu.ca}
\date{}

\renewcommand{\contentsname}{\centering Table of Contents}

\begin{document}
  \maketitle
  \tableofcontents
  \newpage
  \section{Integration Practice} 
  \subsection{FToC I} 
  Recall that we can solve a definite integral using the following definition:
  $$
    \displaystyle\int_{a }^{b} f(x)dx = F(b) - F(a)
  $$ 
  \subsection{Change of Variables / U-Substitution}
  Suppose we need take the antiderivative of $\int 2x \cos(x^2)dx$, let us suppose that $g(x) = x^2$, then we know $g^\prime (x) = 2x$. We also know that $\int \cos (x) dx = \sin(x) + C$. If we combine these, we can derive the answer as:
  $$
  \sin (x^2) + C = \int 2x \cos(x^2)dx
  $$
  \begin{theorem}
    Let us take $u = g^\prime (x) \rightarrow \frac{du }{dx } = g^\prime (x)$ and $du = g^\prime (x) dx$. We can then derive the following:
      \begin{equation}
        \label{Usub}
        \begin{split}
          f(g(x)) &= \int (f \cdot g)^\prime x \\
                  &= \int f^\prime (g(x)) g^\prime(x) dx \\
                  &= \int f^\prime (u) du
        \end{split}
      \end{equation}
  \end{theorem}
  \subsubsection{Examples:}
  Consider the following substitution $u = 3x \rightarrow du = 3x \rightarrow \frac{1}{3} du = dx$, we can then solve:
  \begin{equation}
    \label{Example 1}
    \begin{split}
      \int \cos (3x) dx   
                        &= \frac{1}{3} \int \cos (u) du \\
                        &= \frac{1}{3} \sin (u) + C \\
                        &= \frac{1}{3} \sin (3x) + C
    \end{split}
  \end{equation}
  Consider the following substitution $u = 2x^2 +1 \rightarrow du = 4x\ dx $, we can then solve:
  \begin{equation}
    \label{Example 2}
    \begin{split}
      \int \frac{x }{2x^2+1} dx &= \frac{1}{4} \int \frac{du }{u}\\
                                &= \frac{1}{4} ln(u) + C\\
                                &= \frac{1}{4} ln (2x^2+1) + C
    \end{split}
  \end{equation}
  Consider the following substitution $ u = 1+x \rightarrow du = dx \rightarrow u - 1 = x$
  \begin{equation}
    \label{Example 3}
    \begin{split}
      \int x \sqrt{1+x}\ dx &= \int (u-1) u ^{\frac{1 }{2}} \\
                            &= \int u^{\frac{3 }{2}} - u^{\frac{1 }{2 }}du \\
                            &= \frac{2 }{5} u^{\frac{5 }{2 }} - \frac{2 }{3} u^{\frac{3 }{2}} + C\\
                            &= \frac{2 }{5} (1+x)^{\frac{5 }{2 }} - \frac{2 }{3} (1+x)^{\frac{3 }{2}} + C\\
    \end{split}
  \end{equation}
  \subsection{Area between two curves}
  \textbf{Recall:}

  Suppose $f(x) \geq 0$ is the area beneath the curve $0\le y \le f(x)$, where $a\le x \le b$, then:
  $$
    \displaystyle\int_{a}^{b} f(x) dx  
  $$
  is the area of the curve between $a,b$.

  \begin{theorem}
    Given functions $f(x), g(x)$ the area between the curves from $(a,b)$ can be represented as:
    \begin{displaymath}
     \displaystyle\int_{a }^{b}  (f(x)-g(x))dx
    \end{displaymath}
  \end{theorem}
  \subsubsection{Examples:}
  Given two functions $f(x) = 3x^2 + 12$ and $g(x) = 4x+4$, find the area between the curves from $(-3,3)$.
  \begin{equation}
    \label{Example1}
    \begin{split}
      \displaystyle\int_{-3 }^{3 } (f(x)-g(x))dx &= \displaystyle\int_{-3 }^{3} (3x^2+12) - (4x+4) dx\\
                                               &= \displaystyle\int_{-3 }^{3} 3x^2 - 4x + 8 dx \\
                                               &= x^3-2x^2+8x \|^{3}_{-3} \\
                                               &\vdots\\
                                               &= 102
    \end{split}
  \end{equation}
  Given two functions $(x^2+2)$ and $(2x+5)$, find the \textbf{enclosed} area between these two curves.

 \textit{Find $a,b$, where the lines intesect $\rightarrow 2x+5 = x^2 + 2 \dots (-1 \le x \le 3)$ then solve the integral.} 

  \begin{equation}
    \label{eq:3}
    \begin{split}
      \displaystyle\int_{-1}^{3} (f(x)-g(x)) dx &= \displaystyle\int_{-1 }^{3 } (x^2 + 2)-(2x+5)\\
                                                &\vdots \\
                                                &= \frac{32 }{3}
    \end{split}
  \end{equation}
  Given two functions $\sin(x)$ and $\cos(x)$ find the area between these two curves given that $0\le x \le \frac{\pi }{2}$.
  
  \textit{Find the intersection between the functions on the range given, then build the integral. This is easily done by observing the functions \textbf{geometrically.}}
  \begin{equation}
    \begin{split}
      \displaystyle\int_{0 }^{\frac{\pi }{4}} (f(x)-g(x)) dx &= \displaystyle\int_{0 }^{\frac{\pi }{4}}\cos(x)-\sin(x)dx+\displaystyle\int_{\frac{\pi }{4}}^{\frac{\pi }{2}}\sin(x)-\cos(x)  dx \\ 
                                                             &\vdots\\
                                                             &= 2(\sqrt{2}-1)
    \end{split}
  \end{equation}
  \newpage
  \section{Integration in Theory/Application} 
  \subsection{Volume of Revolution}
  Visualize a cyclindrical shape from $(a,b)$ built by rotating some curve defined by $f(x)$ about some $x$-axis
  \begin{figure}[h]
    \begin{center}
      \includegraphics[width=0.5\textwidth]{img/2023-01-23-20-57-11.png}
    \end{center}
    \caption{\textit{Cylinder built by rotating $f(x)$ about the $x$-axis}}
  \end{figure}

  We can define the volume of this cylinder as the cross section $\times$ thickness, similar to calculating the integral in a 2-dimensional plane. The integral for the volume of this cyclinder is then written as:
  \[
    V = \pi r^2 h 
  \]
  \[
    V =\int_{a }^{b} \pi f(x)^2 dx
  \]

  Assuming another function $g(x)$ also rotated about the same axis, creating a hollow inside of the cylinder, we would then need to subtract the hollow section from the full volume:
  \[
    V = \pi (\underset{outer}{r_o} - \underset{inner}{r_i})^2h
  \]
  \[
    V = \int_{a }^{b} \pi (\underset{outer}{f(x)} - \underset{inner}{g(x)})^2 dx
  \]
  \subsection{The Average Value of a Function}
  Recall that the average value of a set of numbers $(x_1,x_2,\dots,x_n)$ can be written as:
  \[
    Average =\frac{x_1 + x_2 + \dots + x_n}{n}
  \]
  We can then extrapolate to a more generalized formula:
  \[
    Average = \frac{1 }{n} \sum_{i=1}^{n} (x_i)
  \]
  Rewrite in terms of arbitrary even sections between $(a,b)$ on some curve $f(x)$, rather than some set of numbers:
  \[
    \begin{split}
      Average &= \frac{1 }{b-a} \sum_{x=a }^{b} f(x) \quad \textrm{(This can be rewritten as an integral from } (a,b))\\ 
              &= \frac{1 }{b-a} \int_{a }^{b} f(x) dx , \quad a \le x \le b \quad \textrm{(Definition of Average Value of $f(x)$)}
    \end{split}
  \]
  
  \section{Integration Techniques} 
  Techniques of integration can be thought of as methods for evaluating integrals in situations where the process or methodology required are not obvious, many integrals can only be solved by specific techniques and these techniques can generally be categorized into two types:
  \begin{itemize}
    \item Universal Techniques
    \item Particular Techniques
  \end{itemize}
  \subsection{Universal Techniques}
  \subsubsection{Integration by Parts}
  Recall that $(f(x)g(x))^\prime = f(x)g^\prime(x) + f^\prime(x)g(x)$ (\textit{Product Rule}), if we then integrate both sides of this equation:
  \[
    f(x)g(x) = \int f(x)g^\prime(x) dx + \int g(x)f^\prime(x) dx
  \]
  We can then rearrange this equation into the following:
  \[
    \int f(x)g^\prime(x) dx = f(x)g(x) - \int g(x)f^\prime(x) dx
  \]
  Then we can generalize into:
  \[
    \int u dv = uv - \int v du
  \]
  This is known as integration by parts, the idea is that you rewrite the integrand in terms of $u$ $dv$, generally you will want to follow these guidelines for choosing these values:
  \begin{enumerate}
    \item Choose $u$ to be such that $du$ is a \textbf{simpler} value then $u$.
    \item Choose $dv$ such that you could easily integrate and result in $v$.
  \end{enumerate}
  You will know that a integral likely requires integration by parts if the integrand is obviously a product of two functions, take the following example:

  \textit{Evaluate.} $\int x \cos x dx = $
  \begin{equation*}
    \begin{aligned}
      \int u dv &= uv - \int vdu \\
      \int x \cos x dx &= x \sin x - \int \sin x dx \\
                       &= x \sin x + \cos x + C
    \end{aligned}
    \qquad 
    \begin{aligned} 
      u = x &,\quad dv = \cos x dx \\
      du = dx &,\quad v = \int \cos x dx = \sin x
    \end{aligned}
  \end{equation*}
  \subsubsection{Partial Fraction Decomposition}
  When given an integral in the form $\int \frac{P(x)}{Q(x)}dx$ where $P(x),Q(x)$ are some polynomials, you likely will be able to proceed by converting the term into partial fractions. Keep in mind that for this technique to work, the degree of $Q(x)$ must always be \textbf{higher} than $P(x)$ and $Q(x)$ must also be \textbf{factorable}.
  The goal is to take $Q(x)$ into multiple factors, then find some $\frac{a }{q_1(x)} + \frac{b }{q_2(x)} \dots = \frac{P(x)}{Q(x)}$, where $q_n(x)$ are factors of $Q(x)$.

  \textbf{Ex.} Evaluate $\int \frac{1 }{x^2-1} dx$. We can see that $x^2-1 = (x+1)(x-1)$:
  \begin{equation*}
    \begin{aligned}
      & \int \frac{1 }{x^2-1} dx\\
      &= \int \frac{1 }{(x-1)(x+1)} dx\\
      &= \int \frac{-1 }{2} (\frac{1 }{x+1}) + \frac{1 }{2}(\frac{1 }{x-1}) dx \\
      &= \frac{-1 }{2} ln|x+1| + \frac{1 }{2} ln|x-1| + C
    \end{aligned}
    \qquad \Leftarrow \qquad
    \begin{aligned}
      \frac{1 }{x^2-1} &=\frac{a }{x+1} + \frac{b }{x-1} \\
                     1 &=a(x-1) + b(x+1)\\
                       &= ax - a + bx + b \\
       0x + 1                &= x(a+b) + (b-a) \\ 
                       &\rightarrow a = \frac{-1 }{2}, \quad b = \frac{1 }{2}
    \end{aligned}
  \end{equation*}
  \subsection{Particular Techniques}
  \subsubsection{Trigonometric Integrals}
  To evaluate an integral in the form $\int \sin^n x$ or $\int \cos^n x$, where $n$ is some integer, recall the following two trigonometric identities:
  \[
    \begin{split}
      \sin^2 x + \cos^2 x &= 1 \\
      \cos^2 x - \sin^2 x &= \cos 2x \\
    \end{split}
  \]
  Rearranging these identities we can solve for $\sin^2 x$ and $\cos^2 x$:
  \[
    \begin{split}
      \cos^2 x &= \frac{1 }{2} (1+\cos 2x) \\
      \sin^2 x &= \frac{1 }{2} (1-\cos 2x)
    \end{split}
  \]
  Using these equations we can create a general formula for solving trigonometric integrals depending on the parity of $n$.

  \textbf{General formula for when $\int \sin^n / \cos^n x$ when $n$ is even.}

  Take $\int \sin^{2m} x dx$ using the identities; we can solve through substitution.
  \begin{equation}
   \begin{split}
     \int \sin^{2m} x dx&= \int (\sin^2 x)^m dx \\
                      &= \int (\frac{1 }{2}(1-\cos2x))^m dx
   \end{split} 
  \end{equation}
  We can then solve as a trivial integral.

  \textbf{General formula for when $\int \sin^n / \cos^n x$ when $n$ is odd.}

  Take $\sin^{2m+1} xdx$, we can apply a u-sub taking $u = \cos x$ and $du = -\sin x$.
  \begin{equation}
   \begin{split}
     \int \sin^{2m+1}dx &= \int \sin^{2m} x \sin x dx \\
                        &= \int (\sin^2 x)^m \sin x dx \\
                        &= \int (1 - \cos^2 x)^m \sin dx \\
                        &= -\int (1-u^2)^m du
   \end{split} 
  \end{equation}
  If the integral contains both $\sin^n x$ and $\cos^m x$, such as $\int \cos^m x \sin^n x dx$, there are two situations which can occur, and we can solve both using the previous formulae:
  \begin{enumerate}
    \item If \textbf{either} $n$ or $m$ are \textbf{odd}, then simply take apply the odd formula on the function with the odd power, then the u-sub will natually substitute the remaining term.
    \item If \textbf{both} $n$ and $m$ are \textbf{even}, you may apply the even formula to both and expand until you get all values in base terms, this may lead to expanding high degree polynomials.
  \end{enumerate}
  To evaluate an integral in the form $\int \sec^ndx$ or $\int \tan^n x dx$, keep in mind the following useful relations:
  \begin{itemize}
    \item $\frac{d }{dx} \tan x = \sec^2 x$
    \item $\frac{d }{dx} \sec x = \sec x \tan x$
    \item $\int \tan x dx = -ln|\cos x| + C$
    \item $\int \sec x dx = ln(\sec x \tan x) + C$
    \item $\tan^2 x + 1 = \sec^2 x$
  \end{itemize}
  Similarly to integrating $\int \sin^n x/ \cos^n x$, there are different cases to consider regarding whether $n$ is even or odd:
  \begin{enumerate}
    \item The Power of \textbf{Secant} is \textbf{even}, we isolate some $\sec^2$ from the $\sec^n$ term when applicable, then generally we could make a substitution with the $\tan^2 x +1$ relation, then perform a u-sub and solve.
    \item The Power of \textbf{Tangent} is \textbf{odd}, we may try isolating some $\tan^2$ from the $\tan^n$ term when applicable, then it is possible to either make a substitution with the $\sec^2 - 1$ relation, or multiply by $\frac{\sec }{\sec}$ to then make a u-sub.
    \item The Power of \textbf{Secant} is \textbf{odd}, pray that you can reduce into integration by parts, if the degree is $>$3 you may have to do this multiple times.
    \item The Power of \textbf{Tangent} is \textbf{even}, we may isolate into $\tan^2$ from $\tan^n$, then substitute with $\sec^2 - 1$, distribute terms, and either split the integral or integrate directly with a u-sub.
  \end{enumerate}
  \subsubsection{Trigonometric Substitutions}

  The following equations are useful creating simple substitutions. If you can simplify a term to these forms then the substitutions may make the integral much easier to evaluate.

  \begin{itemize}
    \item $\sqrt{a^2-x^2} \rightarrow x =a \sin\theta \quad dx = a\cos\theta d\theta$
    \item $\sqrt{a^2+x^2} \rightarrow x =a \tan\theta \quad dx = a\sec^2\theta d\theta$
    \item $\sqrt{x^2-a^2} \rightarrow x =a \sec\theta \quad dx = a\sec\theta\tan\theta d\theta$

  \end{itemize}
  Keep in mind that after substituting, your integral will be derived with respect to $\theta$ rather than $x$, thus you must relate the final answer back in terms of $x$.

  \section{Infinite limits of Integration} \label{sec:improper integrals}
  Given some improper integral in the forms:
  \begin{itemize}
    \item $\int^\infty_{a} f(x) dx$, we can rewrite as: $\underset{b\rightarrow \infty}{lim}\int^b_a f(x)dx$.
    \item $\int^b_{-\infty} f(x) dx$, we can rewrite as: $\underset{a\rightarrow -\infty}{lim}\int^b_a f(x)dx$.
    \item $\int^\infty_{-\infty} f(x) dx$, we can rewrite as: $\int^c_{-\infty} f(x) dx + \int^\infty_{c} f(x) dx$, for some arbitrary $c$.
  \end{itemize}
  Then we simply evaluate the integral and take the limit. If you can evaluate the integral as it tends to infinity, then you say that the integral \textbf{converges} to some number, otherwise it \textbf{diverges}, meaning it is unevaluatable (\textit{limit does not exist or is infinite}).
  \subsection{Comparison Test for Integrals}
  Given some functions $f(x)$ and $g(x)$, if $0 \le g(x) \le f(x)$ then:

  \begin{itemize}
    \item If $f(x)$ \textbf{converges} $\rightarrow g(x)$ \textbf{converges}.
    \item If $g(x)$ \textbf{diverges} $\rightarrow f(x)$ \textbf{diverges}.
  \end{itemize}

  \section{Sequences \& Series} 
  Given some sequence of numbers in the form:
  \[
    a_1,a_2,a_3,\dots,a_n\ \}\ a_n = f(n)
  \]
  Where we can define some function $f$ to represent every element of $a_n$. We can then use infinite limits of integration~(\ref{sec:improper integrals}) to evaluate whether the sequence of numbers converges. 
  \[
    a_n = f(n) \rightarrow \underset{x\rightarrow \infty}{lim} f(x) = L \rightarrow \underset{n\rightarrow \infty}{lim} a_n = L
  \]
  Given some $\underset{n\rightarrow \infty}{lim} a_n = L$ and $\underset{n\rightarrow \infty}{lim} b_n = K$, the following properties hold:
  \begin{itemize}
    \item $\underset{n\rightarrow \infty}{lim} a_n + b_n = L + K$
    \item $\underset{n\rightarrow \infty}{lim} a_n - b_n = L - K$
    \item $\underset{n\rightarrow \infty}{lim} a_n  b_n = L  K$
    \item $\underset{n\rightarrow \infty}{lim} \frac{a_n}{b_n } = \frac{F }{K}$, when $K \neq 0$
  \end{itemize}
  \subsection{Bounded, Monotonic Series}
  A series $a_n$ is called \textbf{Monotonic} if it is only either \textbf{increasing} or \textbf{decreasing}. We say that $a_n$ is \textbf{bounded above}, if there exists $a_n \le M$ for all n. We say that $a_n$ is \textbf{bounded below} if there exists $N \leq a_n$ for all $n$. If $a_n$ is both bounded above and below then it is referred to as just \textbf{bounded}.
  \subsection{Convergence of Series}
  For a series to converge, means that there is some number that can be evaluated from the sum of some series $\sum_{n=0}^\infty a_n$. For example, a \textbf{Telescoping Series} defined:
  \[
    \sum_{n=0}^\infty \displaystyle\frac{1 }{n(n+1)}
  \]
  Converges to \textbf{1} because it is possible to rewrite the sum using the identity:
  \[
    \displaystyle\frac{1 }{n(n+1)} = \displaystyle\frac{1 }{n} - \displaystyle\frac{1 }{n+1}
  \]
  We can than expand as observe that the terms inbetween the first and last naturally cancel out, since the last term is decreasing and approaches 0, we can take that the sum of this series is \textbf{1}. 
  \subsubsection{Convergence of A Geometric Series}
  Given some $c \neq 0$ and some $|r| < 1$, then the following series \textbf{converges} to:
  \[
    \sum_{n=0}^\infty cr^n = \displaystyle\frac{c }{1-r}
  \]
  If $|r| \geq 1$ then the series \textbf{diverges}.
  \subsubsection{Convergence of p-series}
  The infinite series defined:
  \[
    \sum_{n=0}^\infty \displaystyle\frac{1 }{n^p} 
  \]
  \textbf{Converges} only if $p > 1$ and \textbf{diverges} otherwise.
  \subsection{Absolute and Conditional Convergence}
  \textbf{Absolute Convergence} states that a series $\sum_{n=0}^\infty a_n$ \textbf{converges absolutely} if $\sum_{n=0}^\infty |a_n|$ converges. \newline \hfill
  \noindent\textbf{Conditional Convergence} states that a series $\sum_{n=0}^\infty a_n$ \textbf{converges conditionally} if $\sum_{n=0}^\infty |a_n|$ diverges but $\sum_{n=0}^\infty a_n$ converges.
  \section{Techniques of Evaluating Convergence of Series} 
  \subsection[Divergence Theorem]{$n$th term Divergence Theorem}
  This theorem states that for some series, $\sum_{n=0}^\infty a_n$, if:
   \[
    \lim_{n \rightarrow \infty} a_n \neq 0 
   \] 
   Then the series \textbf{diverges}. However keep in mind that $\lim_{n \rightarrow \infty} a_n = 0$ does \textbf{not} necessarily indicate convergence (ex. $\sum_{n=0}^\infty \displaystyle\frac{1 }{n}$ diverges).
  \subsection{Integral Test}
  Let $a_n = f(n)$, where $f$ is positive, decreasing and continuous, for $x, x\geq 1$. Then the following apply:
  \begin{enumerate}
    \item If $\int_1^\infty f(x)$ \textbf{converges}, then $\sum_{n=1}^\infty a_n$ \textbf{converges}.
    \item If $\int_1^\infty f(x)$ \textbf{diverges}, then $\sum_{n=1}^\infty a_n$ \textbf{diverges}.
  \end{enumerate}
  \subsection{Direct Comparison Test}
  Let $M > 0$ and such that $0 < a_n < b_n$ for $n \geq M$, we can then make the following conclusions:
  \begin{enumerate}
    \item If $\sum_{n=0}^\infty b_n$ \textbf{converges}, then $\sum_{n=0}^\infty a_n$ \textbf{converges}.
    \item If $\sum_{n=0}^\infty a_n$ \textbf{diverges}, then $\sum_{n=0}^\infty a_n$ \textbf{diverges}.
  \end{enumerate}
  \subsection{Limit Comparison Test}
  Let $a_n$ and $b_n$ be positive sequences, assuming the following limit exists:
  \[
    L = \lim_{n\rightarrow \infty} \displaystyle\frac{a_n }{b_n}
  \]
  Then we can make the following conclusions:
  \begin{enumerate}
    \item If $L > 0$, then $\sum_{n=0}^\infty a_n$ \textbf{converges} if and only if $\sum_{n=0}^\infty b_n$ \textbf{converges}.
    \item If $L = \infty$ and $\sum_{n=0}^\infty a_n$ \textbf{converges}, then $\sum_{n=0}^\infty b_n$ \textbf{converges}.
    \item If $L = 0$ and $\sum_{n=0}^\infty b_n$ \textbf{converges}, then $\sum_{n=0}^\infty a_n$ \textbf{converges}.
  \end{enumerate}
  \subsection{Alternating Series Test}
  Take a sequence $b_n$ which is positive, strictly decreasing and converges to 0:
  \[
    \lim_{n\rightarrow \infty} b_n = 0
  \]
  Then the following alternating series also \textbf{converges}:
  \[
    \sum_{n=1}^\infty (-1)^{n-1} b_n = S
  \]
  We can also derive the following:
  \[
    0 < S < b_1 \\ \textrm{ and } S_p < S < S_q \textrm{ for $p$ is even and $q$ is odd.}  
  \]
  \subsection{Ratio Test}
  For a sequence $a_n$, assume the following limit exists:
  \[
    p = \lim_{n\rightarrow \infty} |\displaystyle\frac{a_{n+1}}{a_n}|
  \]
  We can make the following conclusions:
  \begin{enumerate}
    \item If $p<1$, then $\sum a_n$ \textbf{converges absolutely}.
    \item If $p>1$, then $\sum a_n$ \textbf{diverges}.
    \item If $p=1$, then the test is \textbf{inconclusive}.
  \end{enumerate}
  \subsection{Root Test}
  For a sequence $a_n$, assume the following limit exists:
  \[
    L = \lim_{n\rightarrow \infty} \sqrt[n]{|a_n|}
  \]
  \begin{enumerate}
    \item If $p<1$, then $\sum a_n$ \textbf{converges absolutely}.
    \item If $p>1$, then $\sum a_n$ \textbf{diverges}.
    \item If $p=1$, then the test is \textbf{inconclusive}.
  \end{enumerate}
  We can make the following conclusions:
  \section{Power Series} 
  We begin by creating a series of polynomials:
  \[
    \sum_{n=0}^\infty a_n x^n = a^0 + a_1 x^1 + \dots + a_n x^n
  \]
  Specifically, a \textbf{power series} with a center $c$, is defined as:
  \[
    \sum_{n=0}^\infty a_n (x-c)^n = a^0 + a_1 (x-c)^1 + \dots + a_n (x-c)^n = F(x)
  \]
  If for $F(x)$ exists a radius of convergence $R$, for which $|x-c| <R$, then $F(x)$ will converge for all values of $x$, and $F(x)$ will diverge for all values of $x$ when $|x-c| >R$. This means that $F(x)$ will converge for all x in the interval $(c-R, c+R)$, but it the state of the endpoints must be manually evaluated. Note that this value $R$ could be $R=\infty$, meaning $F(x)$ converges for all $x$.
  \subsection{Derivative and Antiderivative of a Power Series}
  For a power series:
  \[
    F(x) = \sum_{n=0}^\infty a_n (x-c)^n
  \]
  which has a radius of convergence $R>0$, then $F$ is differentiable on $(c-R, c+R)$, and we can integrate/differentiate term by term for $x \in (c-R,c+R)$:
  \begin{itemize}
    \item $F^\prime (x) = \sum_{n=0}^\infty n a_n(x-c)^{n-1}$
    \item $\int F (x) = A + \sum_{n=0}^\infty a_n \displaystyle\frac{1}{n+1}(x-c)^{n+1}$, where $A$ is a an arbitrary constant.
  \end{itemize}
  In either case, the radius of convergence $R$, remains the same. However convergence at the endpoints $(c-R, c+R)$, may change.
  \subsection{Manipulation of Power Series}
  \section{Taylor Series} 
  \subsection{Maclaurin Series}
  \subsection{Taylor Series Expansion of Functions}
  \subsection{Taylor Polynomial}
  \subsection{Taylor Theorem}

\end{document}

