\documentclass[a4paper]{article}
\input{pre}

\title{CPS209 Computer Science II}
\author{James Li | 501022159 \and Professor: R. Valenzano\and Email: rick.valenzano@torontomu.ca}
\date{}
\renewcommand{\contentsname}{\centering Content by Week}

\lstset{language = java}

\begin{document}
  \maketitle
  \tableofcontents
  \newpage
  \section{Introduction To Java} 
  \textit{Java is a high-level, class-based, object-oriented programming language that is designed to have as few implementation dependencies as possible. It is a general-purpose programming language intended to let programmers write once, run anywhere}
  \subsection{JVM (Java Virtual Machine)}
  The JVM is the virtual environment in which all Java code can be executed, to run a Java file using the JVM it must first be compiled. The compiler (javac) generates byte code in a $.class$ file which can run on any JVM, allowing cross platform accessibility. The JVM efficiently interprets byte code in the $.class$ file into native binary and executes it, leading to faster processing times than languages like python.
  \subsection{"Hello World!" in Java}
  Begin by creating a file called $HelloWorld.java$ and write the following code:
  \begin{lstlisting}
  public class HelloWorld{
    public static void main(String[] args){
      System.out.println("Hello World!");
    }
  }
  \end{lstlisting}
  Notice the increased verbosity compared to Python, this is a defining trait of Java.
  The code is then compiled using the command:
  \begin{lstlisting}
    >javac HelloWorld.java
  \end{lstlisting}
  This will create a HelloWorld.class file, which we can finally run by invoking the command:
  \begin{lstlisting}
    >java HelloWorld
    Hello World!
  \end{lstlisting}
  \subsection{Syntactic differences with Python}
  \begin{itemize}
    \item Instead of blocks seperated using indents, Java relies on $\{code\}$ to seperate blocks and levels of code.
    \item You must add a $;$ to the end of a line of code, otherwise Java will continue reading all the code afterwards as one line.
  \end{itemize}
  
  \section{Dont need notes class is free} 
  
\end{document}
